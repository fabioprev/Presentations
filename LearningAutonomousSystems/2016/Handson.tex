\section{Handson}

\begin{frame}
	\frametitle{}
	
	\Huge
	
	\vspace{0.5cm}
	
	\begin{center}
		\textbf{PTracking}
	\end{center}
\end{frame}

\begin{frame}
	\frametitle{Hands-on PTracking}
	\framesubtitle{Getting Library}
	
	\Large
	
	\emph{PTracking} is currently available at the following repo:
	\begin{center}
		\url{https://fabioprev@bitbucket.org/fabioprev/ptracking.git}
	\end{center}
	
	\vspace{0.2cm}
	
	Currently the only supported platform is \textbf{Linux}.
\end{frame}

\begin{frame}
	\frametitle{Hands-on PTracking}
	\framesubtitle{Dependencies}
	
	\Large
	
	On Xubuntu (Ubuntu) 14.04 LTS (kernel 3.13.0-37) and later versions, you need to install
	the following packages:
	
	\vspace{0.2cm}
	
	\begin{columns}[T]
		\column{.5\textwidth}
		
		\begin{itemize}
			\item build-essential
			\item cmake
			\item libxml2
			\item libxml2-dev
			\item libboost1.54-all-dev
		\end{itemize}
		
		\column{.5\textwidth}
		\centering
		
		\begin{itemize}
			\item libopencv-dev
			\item gnuplot
			\item gnuplot-x11
			\item libopenni2-dev (optional)
			\item Imbs (see next slide)
		\end{itemize}
	\end{columns}
\end{frame}

\begin{frame}
	\frametitle{Hands-on PTracking}
	\framesubtitle{Getting, Building and Installing Imbs}
	
	\Large
	
	\vspace{0.4cm}
	
	\emph{Imbs} is currently available at the following repo:
	\vspace{-0.2cm}
	\begin{center}
		\url{https://github.com/fabioprev/Imbs.git}
	\end{center}
	
	\vspace{-0.2cm}
	
	\begin{itemize}
		\item \texttt{cd <Imbs-root-directory>}
		\item \texttt{mkdir build}
		\item \texttt{cd build}
		\item \texttt{cmake ../src}
		\item \texttt{make -j<number-of-cores+1>}
		\item \texttt{sudo make install}
	\end{itemize}
\end{frame}

\begin{frame}
	\frametitle{Hands-on PTracking}
	\framesubtitle{Building Library}
	
	\Large
	
	We recommend a so-called out of source building, which can be achieved by the following command
	sequence:
	
	\vspace{0.2cm}
	
	\begin{itemize}
		\item \texttt{cd <PTracking-root-directory>}
		\item \texttt{mkdir build}
		\item \texttt{cd build}
		\item \texttt{cmake ../src}
		\item \texttt{make -j<number-of-cores+1>}
	\end{itemize}
\end{frame}

\begin{frame}
	\frametitle{Hands-on PTracking}
	\framesubtitle{Installing Library}
	
	\Large
	
	\vspace{0.4cm}
	
	The library can be optionally installed (which we suggest) by typing the following command
	sequence:
	
	\vspace{0.2cm}
	
	\begin{itemize}
		\item \texttt{cd <PTracking-root-directory>/build}
		\item \texttt{sudo make install}
	\end{itemize}
	
	\vspace{0.3cm}
	
	\textbf{Header files:} \texttt{/usr/local/include/PTracking} \\
	
	\vspace{0.1cm}
	
	\textbf{Shared objects:} \texttt{/usr/local/lib/PTracking} \\
	
	\vspace{0.1cm}
	
	\textbf{Binaries:} \texttt{/usr/local/bin} \\
	
	\vspace{0.5cm}
	
	\textbf{Warning:} First logout before starting using the library
\end{frame}

\begin{frame}
	\frametitle{Hands-on PTracking}
	\framesubtitle{Running Library}
	
	\Large
	
	Two examples are provided along the \emph{PTracking} library:
	
	\begin{itemize}
		\item \texttt{cd <PTracking-root-directory>/script}
		\item \texttt{./run\_Edinburgh-Atrium.sh} or
		\item \texttt{./run\_PETS-2009.sh}
	\end{itemize}
\end{frame}

\begin{frame}
	\frametitle{Projects}
	
	\LARGE
	
	\begin{block}{Project \# 1}
		Study \textbf{how} the tracking parameters - for example the \emph{closenessThreshold} -
		\textbf{influence} the data association process. The analyse should accurately report the
		\textbf{variation} - in terms of MOTA and MOTP - of the performance while \textbf{varying} the
		tracking parameters.
	\end{block}
\end{frame}

\begin{frame}
	\frametitle{Projects}
	
	\LARGE
	
	\begin{block}{Project \# 2}
		Study \textbf{how} the sensor calibration \textbf{influences} the tracking performance. The
		analyse should accurately report the \textbf{variation} - in terms of MOTA and MOTP - of the
		performance while \textbf{varying} the precision in the calibration phase.
	\end{block}
\end{frame}
