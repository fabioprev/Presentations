\section{Related Work}

\begin{frame}
	\frametitle{Related Work}
	\framesubtitle{Multi-object tracking}
	
	\Large
	
	\vspace{0.4cm}
	
	Multi-object tracking algorithms can be classified into two groups \cite{Andriyenko11}:
	
	\vspace{0.2cm}
	
	\begin{itemize}
		\item \textbf{Global} (or \emph{offline}): formulating the tracking problem as an
			  optimization one, where all the trajectories within a temporal window are
			  optimized jointly
		\item \textbf{Recursive} (or \emph{online}): estimating the current state relying
			  only on the current observations and on the previous state
	\end{itemize}
\end{frame}

\begin{frame}
	\frametitle{Multi-Object Tracking}
	\framesubtitle{Global methods}
	
	\Large
	
	\vspace{0.2cm}
	
	\begin{itemize}
		\item \textbf{Berclaz} \emph{et al.} \cite{Berclaz11}: mathematically sound
			  multiple object tracking framework based on a k-shortest path optimization
			  algorithm
		\vspace{0.1cm}
		\item \textbf{Leal-Taix{\'e}} \emph{et al.} \cite{Leal11}: formulate a new graph
			  model for the multiple object tracking challenge by minimizing a network flow
			  problem
		\vspace{0.1cm}
		\item \textbf{Sharma} \emph{et al.} \cite{Sharma09}: adapting a Cluster-Boosted
			  Tree based pedestrian detector to deal with the people tracking problem
		\vspace{0.1cm}
		\item ...
	\end{itemize}
\end{frame}

\begin{frame}
	\frametitle{Multi-Object Tracking}
	\framesubtitle{Recursive methods}
	
	\Large
	
	\vspace{0.2cm}
	
	\begin{itemize}
		\item \textbf{Breitenstein} \emph{et al.} \cite{Breitenstein11}: online method for
			  multi-person tracking-by-detection in a particle filtering framework
		\vspace{0.1cm}
		\item \textbf{Yang} \emph{et al.} \cite{Yang09}: probabilistic appearance model
			  method for tracking multiple people
		\vspace{0.1cm}
		\item ...
	\end{itemize}
\end{frame}

\begin{frame}
	\frametitle{Global vs Recursive Methods}
	
	\begin{table}[!t]
		\centering
		\begin{tabular}{ c | c | c | }
			\cline{2-3}
			& \textbf{Global Methods} & \textbf{Recursive Methods} \\ \hline
			
			\multicolumn{1}{|c|}{\textbf{Accuracy}} & medium/high & medium/high \\ \hline
			\multicolumn{1}{|c|}{\textbf{Precision}} & \textbf{high} & medium/high \\ \hline
			\multicolumn{1}{|c|}{\textbf{Robustness}} & \textbf{high} & medium/high \\ \hline
			\multicolumn{1}{|c|}{\textbf{Computational Load}} & high & \textbf{low}/medium \\ \hline
			\multicolumn{1}{|c|}{\textbf{Real-time}} & no & \textbf{yes}/no \\ \hline
		\end{tabular}
	\end{table}
	
	\vspace{0.4cm}
	
	Global methods are more precise and more robust but, on the other hand, they
	cannot run in a real system because:
	
	\begin{itemize}
		\item no information from the future are available
		\item frame rate not suitable for real-time applications
	\end{itemize}
\end{frame}
