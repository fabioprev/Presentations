\section{Experimental Evaluation - Activity Forecasting}

\logo
{
	\begin{tikzpicture}
		\hspace{0.09cm}
		\node at (-1,0) [draw=white,ultra thick,inner sep=0pt]
		{
			\includegraphics[height=1cm]{ThemeFigs/Sapienza}
		};
		\node at (0,0) [draw=white,ultra thick,inner sep=0pt]
		{
			\includegraphics[height=1cm]{ThemeFigs/Edinburgh}
		};
	\end{tikzpicture}
	\vspace{199.1pt}
}

\begin{frame}
	\frametitle{Evaluating an Activity Forecasting Method}
	
	\Large
	
	\vspace{0.45cm}
	
	The Negative Log-Loss (NLL) represents the expectation of the log-likelihood of a trajectory $ s $
	under a policy $ \pi(a \, | \, s) $:
	\vspace{-0.1cm}
	\begin{equation*}
		NLL(s) = E_{\pi(a \, | \, s)} \Big [ -\log \prod\nolimits_t \pi(a_t \, | \, s_t)  \Big ]
	\end{equation*}
	
	\vspace{0.1cm}
	
	This metric measures the probability of drawing the demonstrated trajectory from the learnt
	distribution over all possible trajectories. \\
\end{frame}

\begin{frame}
	\frametitle{Experimental Evaluation}
	\framesubtitle{Activity Forecasting}
	
	\large
	
	\vspace{-0.1cm}
	
	\begin{columns}[t]
		\only<1->
		{
			\column{0.9\textwidth}
			
			\begin{block}{Predicting Future Agent Motions}
				inferring agent goals accurately in a varied set of environments
			\end{block}
			
			\column{0.05\textwidth}
		}
	\end{columns}
	
	\vspace{0.1cm}
	
	\begin{center}
		\begin{tikzpicture}
			\node at (0,0) [draw=black,ultra thick,inner sep=0pt]  {\includegraphics[height=2.9cm]{Figures/InformaticsForum}};
			\node at (4,0) [draw=black,ultra thick,inner sep=0pt]  {\includegraphics[height=2.9cm]{Figures/InSpace_FrontCamera}};
			\node at (7.98,0) [draw=black,ultra thick,inner sep=0pt]  {\includegraphics[height=2.9cm]{Figures/VIRAT}};
		\end{tikzpicture}
	\end{center}
	
	\vspace{0.15cm}
	
	\tiny
	
	\textbf{IRL-based Prediction of Goals for Dynamic Environments}\\
	F. Previtali, A. Bordallo, S. Ramamoorthy \\
	\emph{Workshop on Machine Learning for Social Robotics at ICRA, 2015} \\
	
	\vspace{0.1cm}
	
	\textbf{Predicting Future Agent Motions for Dynamic Environments}\\
	F. Previtali, A. Bordallo, L. Iocchi, S. Ramamoorthy \\
	\emph{International Conference on Intelligent Robots and Systems} [submitted] \\
\end{frame}

\begin{frame}
	\frametitle{Quantitative Evaluation}
	\framesubtitle{Activity Forecasting}
	
	\begin{figure}
		\centering
		\begin{tikzpicture}
			\node at (0,0) [draw=white,ultra thick,inner sep=0pt]
			{
				\includegraphics[width=\linewidth]{Figures/QuantitativeEvaluation}
			};
		\end{tikzpicture}
	\end{figure}
\end{frame}

\begin{frame}
	\frametitle{Experimental Evaluation}
	\framesubtitle{Context-Aware Navigation}
	
	\vspace{0.05cm}
	
	\begin{center}
		\begin{tikzpicture}
			\node at (0,0) [draw=black,ultra thick,inner sep=0pt]
			{
				\includegraphics[height=2.6cm]{Figures/Kinect-1}
			};
			\node at (3.62,0) [draw=black,ultra thick,inner sep=0pt]
			{
				\includegraphics[height=2.6cm]{Figures/Kinect-2}
			};
			\node at (7.24,0) [draw=black,ultra thick,inner sep=0pt]
			{
				\includegraphics[height=2.6cm]{Figures/Kinect-3}
			};
		\end{tikzpicture}
	\end{center}
	
	\vspace{-0.4cm}
	\tiny
	
	\begin{tabbing}
		\hspace{1.5cm}
		\textbf{Counterfactual Reasoning about Intent for Interactive Navigation in Dynamic
				Environments} \\
		\hspace{1.5cm}
		A. Bordallo, F. Previtali, N. Nardelli, S. Ramamoorthy \\
		\hspace{1.5cm}
		\emph{IEEE/RSJ International Conference on Intelligent Robots and Systems, 2015} \\
	\end{tabbing}
	
	\vspace{-0.55cm}
	
	\begin{center}
		\begin{tikzpicture}
			\node at (0,0) [draw=black,ultra thick,inner sep=0pt]
			{
				\includegraphics[height=2.35cm]{Figures/ConstantVelocityComparison}
			};
			\node at (3.45,0) [draw=black,ultra thick,inner sep=0pt]
			{
				\includegraphics[height=2.35cm]{Figures/ProxemicsComparison}
			};
			\node at (6.9,0) [draw=black,ultra thick,inner sep=0pt]
			{
				\includegraphics[height=2.35cm]{Figures/InteractiveCostmapComparison}
			};
		\end{tikzpicture}
	\end{center}
	
	\vspace{-0.4cm}
	
	\begin{tabbing}
		\hspace{1.55cm}
		\textbf{Interactive Costmaps: Integrating Prediction and Planning with Counterfactual
				Reasoning} \\
		\hspace{1.55cm}
		A. Bordallo, F. Previtali, S. Ramamoorthy \\
		\hspace{1.55cm}
		\emph{IEEE/RSJ International Conference on Intelligent Robots and Systems} [submitted] \\
	\end{tabbing}
\end{frame}

\begin{frame}
	\frametitle{Qualitative Evaluation}
	\framesubtitle{InSpace}
	
	\begin{figure}[!h]
		\centering
		\includemovie[inline=false,text=
		{
			\begin{tikzpicture}
				\node at (0,0) [draw=black,ultra thick,inner sep=0pt]
				{
					\includegraphics[width=0.65\linewidth]{Figures/InSpace}
				};
			\end{tikzpicture}
		}]{}{}{../Videos/3-Youbot/PTracking-2People_3PlanningYoubots.mpg}
	\end{figure}
\end{frame}
