\documentclass{beamer}

\mode<presentation>
{
	\usepackage{StyleFiles/Rome}
	\setbeamercovered{transparent}
}

\mode<handout>
{
	\usepackage{pgfpages}
	\pgfpagesuselayout{2 on 1}[a4paper,border shrink=5mm]
	\nofiles
}

\usepackage[english]{babel}
\usepackage[latin1]{inputenc}
\usepackage{dsfont}
\usepackage{setspace}
\usepackage[algoruled]{algorithm2e}
\usepackage{algorithmic,algorithm2e,float}

\SetAlCapFnt{\scriptsize}
\SetAlCapNameFnt{\scriptsize}

\setbeamertemplate{itemize subitem}{\tiny\raise1.5pt\hbox{\donotcoloroutermaths$ \blacktriangleright $}}

\title[Automatically classifying Alzheimer's disease by MRI analysis]{\huge Automatically classifying Alzheimer's disease by MRI analysis}

\subtitle{}

\author[Fabio Previtali]{\vspace{0.1cm} \\ \large Fabio Previtali \\ Paola Bertolazzi \\ Giovanni Felici \\ Emanuel Weitschek}

\date[\today]{}

\begin{document}

\begin{frame}[plain]
	\begin{center}
		\LARGE
		
		\vspace{0.5cm}
		
		\textbf{A novel method and software for automatically classifying Alzheimer's disease
		patients by magnetic resonance imaging analysis}
		
		\vspace{0.7cm}
		
		\large
		
		\emph{F. Previtali}, \emph{P. Bertolazzi}, \emph{G. Felici} and \emph{E. Weitschek}
		
		\vspace{1cm}
		
		\normalsize
		
		Workshop/Conference on $ \ldots $ \\
		$ \langle $place$ \rangle $ \\
		$ \langle $date$ \rangle $
		
		\vspace{0.6cm}
		
		\begin{tikzpicture}
			\node at (0,0) [draw=white,ultra thick,inner sep=0pt]
			{
				\includegraphics[width=0.45\linewidth]{ThemeFigs/CNR}
			};
			\node at (6,0) [draw=white,ultra thick,inner sep=0pt]
			{
				\includegraphics[width=0.45\linewidth]{ThemeFigs/Uninettuno}
			};
		\end{tikzpicture}
	\end{center}
\end{frame}

\logo
{
	\begin{tikzpicture}
		\hspace{0.09cm}
		\node at (-1.2,0) [draw=white,ultra thick,inner sep=0pt]
		{
			\includegraphics[height=1cm]{ThemeFigs/CNR-Logo}
		};
		\node at (0,0) [draw=white,ultra thick,inner sep=0pt]
		{
			\includegraphics[height=1cm]{ThemeFigs/Uninettuno-Logo}
		};
	\end{tikzpicture}
	\vspace{199.1pt}
}

\section{Introduction to Distributed Data Fusion}

\begin{frame}
	\frametitle{Multi-Agent Multi-Object Tracking (MAMOT)}
	\framesubtitle{A brief overview}
	
	\begin{columns}
		\column{.7\textwidth}
		\centering
		
		\begin{block}{Object Tracking}
			Detect objects and follow their trajectories
		\end{block}
		
		\begin{block}{MAMOT}
			A team of agents estimate the object trajectories using sensors with a limited field-of-view
		\end{block}
		
		\begin{itemize}
			\item distributed environment (network connected agents)
			\item best-effort communication channel
		\end{itemize}
		
		\column{.28\textwidth}
		\centering
		
		\begin{tikzpicture}
			\node at (0,0) [draw=black,ultra thick,inner sep=0pt]  {\includegraphics[width=3cm]{Figures/Player.png}};
		\end{tikzpicture}
	\end{columns}
\end{frame}

\begin{frame}
	\frametitle{Issues on DDF}
	\framesubtitle{The Data Association problem}
	
	\begin{columns}[T]
		\column{.55\textwidth}
		
		\begin{itemize}
			\only<1>{\item In MAMOT, the pool of gathered parameters are used to compute the global estimation}
			\only<1>{\item Using a Particle Filter, one of the main problem that arises is how to avoid poor estimation
						   quality}
			\only<1>{\vspace{3.42cm}}
			\only<2->{\item The weight of particles is evaluated using the received parameters}
			\only<3->{\vspace{1.5cm}\item[1] Using all parameters influences the whole distribution}
			\only<4-5>{\vspace{1.1cm}}
			\only<6>{\vspace{0.9cm}}
			\only<4>{\item[2] We can assign better weights by use clustering before weighting}
			\only<5>{\item[2] The parameters are associated to the clusters}
			\only<6>{\item[2] The weights of particles in a cluster is given by the parameters associated to it}
		\end{itemize}
		
		\column{.45\textwidth}
		\centering
		
		\begin{tikzpicture}
			\only<1>
			{
				\node at (0,0) [draw=white,thick,inner sep=0pt]  {\includegraphics[width=5cm]{Figures/MamotDDF.pdf}};
			}
			\only<2->
			{
				\node at (-1.3,0) [draw=white,thick,inner sep=0pt]  {\includegraphics[width=2.5cm]{Figures/Issue0.pdf}};
				\node at (1.3,0) [draw=white,thick,inner sep=0pt]  {\includegraphics[width=2.5cm]{Figures/Issue0Bis.pdf}};
				\node at (0,-1) [inner sep=0pt]  {\includegraphics[width=5cm]{Figures/Parentheses.pdf}};
				\node at (0,-2.5) [draw=white,thick,inner sep=0pt]  {\includegraphics[width=4cm]{Figures/IssueBlank.pdf}};
				\node at (0,-4.8) [draw=white,thick,inner sep=0pt]  {\includegraphics[width=4cm]{Figures/IssueBlank.pdf}};
			}
			\only<3->
			{
				\node at (0,-2.5) [draw=white,thick,inner sep=0pt]  {\includegraphics[width=4cm]{Figures/Issue1.pdf}};
				\node at (0,-4.8) [draw=white,thick,inner sep=0pt]  {\includegraphics[width=4cm]{Figures/IssueBlank.pdf}};
			}
			\only<4>
			{
				\node at (0,-4.8) [draw=white,thick,inner sep=0pt]  {\includegraphics[width=4cm]{Figures/Issue2.pdf}};
			}
			\only<5>
			{
				\node at (0,-4.8) [draw=white,thick,inner sep=0pt]  {\includegraphics[width=4cm]{Figures/Issue2Bis.pdf}};
			}
			\only<6>
			{
				\node at (0,-4.8) [draw=white,thick,inner sep=0pt]  {\includegraphics[width=4cm]{Figures/Issue2Ter.pdf}};
			}
		\end{tikzpicture}
	\end{columns}
\end{frame}

\section{Related Work}

\begin{frame}
	\frametitle{Related Work}
	
	\Large
	
	\vspace{0.6cm}
	
	Magnetic resonance imaging does have a good ability for differentiating between soft tissues like
	the brain.
	
	\vspace{0.2cm}
	
	Multiple approaches successfully used MRI brain scans for abnormality/normality Alzheimer
	assessment \cite{DeVos16,Devanand07,Wolz11}.
	
	\vspace{0.2cm}
	
	Many other imaging modalities exists in order to study Alzheimer and other brain diseases such as
	computed tomography (CT) and positron emission tomography (PET).
	
	\vspace{0.2cm}
	
	\tiny
	
	\cite{DeVos16} F. de Vos \emph{et al.}, ``Combining multiple anatomical MRI measures improves
	Alzheimer's disease classification'', Human Brain\\ \hspace{0.25cm} Mapping, 2016
	
	\cite{Devanand07} D. P. Devanand \emph{et al.}, ``Hippocampal and entorhinal atrophy in mild
	cognitive impairment prediction of Alzheimer disease'',\\ \hspace{0.25cm} Journal on Neurology,
	2007
	
	\cite{Wolz11} R. Wolz \emph{et al.}, ``Multi-method analysis of MRI images in early diagnostics of
	Alzheimer's disease'', PloS One, 2011
\end{frame}

\begin{frame}
	\frametitle{Related Work}
	\framesubtitle{Magnetic Resonance Imaging}
	
	\Large
	
	\vspace{0.67cm}
	
	We \emph{discuss} methods that are mostly related with MRI.
	
	\vspace{0.3cm}
	
	Anandh \emph{et al.} \cite{Anandh16} propose to \textbf{differentiate} Mild Cognitive Impairment
	(MCI), Condition Normal (CN) and AD subjects from MRI scans using an approach based on
	\textbf{Laplace-Beltrami eigenvalue} shape descriptors. Those descriptors are identified and their
	performance is \textbf{analysed} using linear Support Vector Machine (SVM) classifier.
	
	\vspace{0.6cm}
	
	\tiny
	
	\cite{Anandh16} K. R. Anandh \emph{et al.}, ``A method to differentiate mild cognitive impairment
	and Alzheimer in MR images using eigen value\\ \hspace{0.25cm} descriptors'', Journal on Medical
	Systems, 2016
\end{frame}

\begin{frame}
	\frametitle{Related Work}
	
	\Large
	
	\vspace{1cm}
	
	\begin{itemize}
		\item Beheshti \emph{et al.} \cite{Beheshti16} describe the use of \textbf{t-test} based
			  feature-ranking approach as part of their feature extraction procedure, where the number
			  of \textbf{top features} is determined using the Fisher criterion
		\vspace{0.2cm}
		\item Plocharski \emph{et al.} \cite{Plocharski16} design a technique to extract sulcal features
			  by means of computing a sulcal medial surface for AD/CN classification
	\end{itemize}
	
	\vspace{0.7cm}
	
	\tiny
	
	\cite{Beheshti16} I. Beheshti \emph{et al.}, ``Feature-ranking-based Alzheimer's disease
	classification from structural MRI'', Journal on Magnetic\\ \hspace{0.25cm} Resonance Imaging, 2016
	
	\cite{Plocharski16} M. Plocharski \emph{et al.}, ``Extraction of sulcal medial surface and
	classification of Alzheimer's disease using sulcal features'',\\ \hspace{0.25cm} Journal on
	Computer Methods and Programs in Biomedicine, 2016
\end{frame}

%\section{Method}

\begin{frame}
	\frametitle{Contributions}
	
	\Large
	
	\vspace{0.8cm}
	
	We present an automated approach for classifying Alzheimer's disease patients from MRI brain scans.
	
	\begin{enumerate}
		\item key points extracted with a \textbf{recent} feature extraction technique, called ORB
			  \cite{Rublee11}
		\item final set of features obtained by \textbf{defining} two \textbf{new} metrics:
			  \textbf{spatial position} of extracted key points and \textbf{their distribution} around
			  the patient's brain
		\item \textbf{fast} and \textbf{reliable} approach for a straightforward deploy in clinical
			  applications
	\end{enumerate}
	
	\vspace{0.58cm}
	
	\tiny
	
	\cite{Rublee11} E. Rublee \emph{et al.}, ``ORB: an efficient alternative to SIFT or SURF'',
	International Conference on Computer Vision, 2011
\end{frame}

\begin{frame}
	\frametitle{Methodology}
	
	\Large
	
	\vspace{0.7cm}
	
	We focus on extracting features which are \textbf{key points} found in the image. In particular, we
	choose as features a set of key points from the MRI scan \textbf{computed} with the ORB feature
	extractor, and we \textbf{associate} to them their \textbf{spatial position} and
	\textbf{distribution} around the patient's brain.
	
	\begin{center}
		\begin{tikzpicture}
			\node at (-5.5,0) [draw=white,ultra thick,inner sep=0pt]
			{
				\includegraphics[height=2.5cm]{Figures/CantorHash}
			};
			\node at (0,0) [draw=white,ultra thick,inner sep=0pt]
			{
				\includegraphics[height=2.5cm]{Figures/Histograms}
			};
		\end{tikzpicture}
	\end{center}
\end{frame}

\begin{frame}
	\frametitle{Methodology}
	\framesubtitle{Key point features}
	
	\Large
	
	\vspace{0.7cm}
	
	Many different approaches have been proposed for feature extraction. We focus on a
	\textbf{computationally-efficient} technique - called ORB - which is based on a FAST key point
	detector. It provides \textbf{good matching}, \textbf{slighlty} affected by image noise and
	\textbf{real-time} performance.
	
	\vspace{0.1cm}
	
	\begin{center}
		\begin{tikzpicture}
			\node at (-2.6,0) [draw=red,ultra thick,inner sep=0pt]
			{
				\includegraphics[height=2.5cm]{Figures/OrbKeypoint}
			};
			\node at (0,0) [draw=red,ultra thick,inner sep=0pt]
			{
				\includegraphics[height=2.5cm]{Figures/OrbDescriptor}
			};
		\end{tikzpicture}
	\end{center}
\end{frame}

\begin{frame}
	\frametitle{Methodology}
	\framesubtitle{Key point features}
	
	\Large
	
	\begin{columns}[T]
		\column{0.35\textwidth}
		
		\begin{center}
			\begin{tikzpicture}
				\node at (0,0) [draw=red,ultra thick,inner sep=0pt]
				{
					\includegraphics[height=3.5cm]{Figures/OrbDescriptor}
				};
			\end{tikzpicture}
		\end{center}
		
		\column{0.58\textwidth}
		
		\vspace{0.65cm}
		
		The feature descriptor of each key point is a \textbf{32-vector} of the pixel
		\textbf{intensity}. The image is described by a $ k \times 32 $ matrix, where $ k $ represents
		the number of key points found in the image.
		
		\column{0.05\textwidth}
	\end{columns}
\end{frame}

\begin{frame}
	\frametitle{Methodology}
	\framesubtitle{Feature extraction}
	
	\Large
	
	\vspace{0.4cm}
	
	We investigate different strategies for building the feature matrix based on the key points
	extracted from the MRI scans and their properties:
	
	\begin{enumerate}
		\item Image descriptor
		\item Histograms
		\item Cantor hash
		\item Histograms and Cantor hash
	\end{enumerate}
\end{frame}

\begin{frame}
	\frametitle{Methodology}
	\framesubtitle{Image descriptor}
	
	\Large
	
	\vspace{0.5cm}
	
	The image descriptor is built by \textbf{straightforwardly} considering the $ k \times 32 $ matrix
	representing the ORB descriptors associated to $ k $ key points of a MRI scan. Each element of the
	matrix is an integer within $ [0,255] $
	
	\begin{center}
		\begin{tikzpicture}
			\node at (0,0) [draw=red,ultra thick,inner sep=0pt]
			{
				\includegraphics[height=3.1cm]{Figures/OrbDescriptor}
			};
		\end{tikzpicture}
	\end{center}
\end{frame}

\begin{frame}
	\frametitle{Methodology}
	\framesubtitle{Histograms}
	
	\Large
	
	\vspace{0.3cm}
	
	Key points \textbf{spatial distribution} is obtained by:
	
	\vspace{-0.1cm}
	
	\begin{itemize}
		\item \textbf{dividing} the MRI scan in a grid $ G = r \times c $ 
		\item for each cell of the grid we compute a \textbf{histogram} representing the number of key
			  points belonging to such a cell
	\end{itemize}
	
	In this way, we build a feature vector by providing the number of key points belonging to every
	cell of the grid. The histogram value for each cell is bounded between $ [0,k] $ and the $
	\sum_{i=1}^r \sum_{j=1}^c G(i,j) = k $
\end{frame}

\begin{frame}
	\frametitle{Methodology}
	\framesubtitle{Cantor hash}
	
	\Large
	
	\vspace{0.3cm}
	
	Key points \textbf{spatial position} is obtained by computing the \emph{Cantor} hash for each key
	point in order to \textbf{map} its Cartesian position into a mono-dimensional space.
	
	\vspace{0.2cm}
	
	In this case, the feature vector contains the hash value $ h(x,y) $ of all key points. The
	Cartesian coordinates $ x $ and $ y $ are both 8-bit since they represent grey-scale pixel
	
	\begin{center}
		\begin{tikzpicture}
			\node at (0,0) [draw=white,ultra thick,inner sep=0pt]
			{
				\includegraphics[height=2.5cm]{Figures/CantorHash}
			};
		\end{tikzpicture}
	\end{center}
\end{frame}

\begin{frame}
	\frametitle{Methodology}
	\framesubtitle{Histograms and Cantor hash}
	
	\Large
	
	\vspace{0.3cm}
	
	We decide to use also the \textbf{combination} of histograms and Cantor hashing for building the
	feature matrix considering \textbf{at the same time} the spatial position of key points as well as
	their distribution around the patient brain.
	
	\vspace{0.3cm}
	
	The Cantor hashing function provides information on the \textbf{exact position} of each key point
	in the brain whilst in the histogram, we use the position of the key points to approximate the
	\textbf{geometrical shape} of them around the brain
\end{frame}

\begin{frame}
	\frametitle{Methodology}
	\framesubtitle{Feature matrix}
	
	\Large
	
	\vspace{0.3cm}
	
	We collect $ n $ brain patients' scan, each one with its $ m $ features and their class label
	(e.g., AD, MCI, LMCI, CN). Every patient's scan $ i \in [1,n] $ with its $j \in [1,m] $ features is
	represented by the vector
	
	\vspace{-0.5cm}
	
	\begin{equation*}
		f_i =(f_{i1}, f_{i2}, \cdots, f_{ij}, \cdots, f_{im}, f_{ic}) \;\;\;\;\;\; 
	\end{equation*}

	where $ f_{ij} \in \mathbb{Z} $ and $ f_{ic} \in {\{AD, MCI, LMCI, CN\}} $.
	
	\vspace{0.3cm}
	
	The \textbf{data matrix} is obtained by the \textbf{union of vectors} $ f_1, f_2, \cdots, f_n $,
	where rows represent the MRI patients' scan and columns the features
\end{frame}

\begin{frame}
	\frametitle{Methodology}
	\framesubtitle{Classification}
	
	\Large
	
	\vspace{0.3cm}
	
	We use Support Vector Machines (SVMs) which is a very \textbf{successful} approach for classifying
	data especially in \textbf{high dimensional} feature spaces.
	
	\vspace{0.3cm}
	
	We choose SVMs, because of their \textbf{proven success} in imaging classification and their
	\textbf{good generalisation ability}. This property is obtained by maximising the margin between
	the closest points to the hyperplane (called support vectors) and the hyperplane defining the
	boundaries of different classes
\end{frame}

%\section{Experiments}

\begin{frame}
	\frametitle{Quantitative Evaluation}
	
	
\end{frame}

%\section{Conclusions}

\begin{frame}
	\frametitle{Conclusions}
	
	\Large
	
	\vspace{0.3cm}
	
	We presented:
	
	\begin{enumerate}
		\item \emph{PTracking}, a novel \textbf{real-time} distributed tracker
		\item \textbf{Effective} framework for \textbf{predicting} future agent motions of
			  goal-oriented agents
	\end{enumerate}
	
	\vspace{0.2cm}
	
	\underline{\textbf{Main contributions}}
	
	\begin{itemize}
		\item Real-time, accurate and precise tracking
		\item Fully scalable design
		\item Prediction without prior annotation of scene semantics
		\item Non-uniform grids for state representation
		\item Efficient and scalable for on-robot implementation
	\end{itemize}
\end{frame}

\begin{frame}
	\frametitle{Future Work}
	
	\Large
	
	Possible future directions in terms of tracking could be:
	\begin{itemize}
		\item \textbf{Integration} of recent real-time \textbf{feature extractors} in the data
			  association module
		\item \textbf{Generation} of a \textbf{3D model} by merging information coming from multiple
			  sources
	\end{itemize}
	
	while, in terms of prediction of future agent motions could be the employment of \textbf{more
	sophisticated} \emph{IRL} alternatives within the overall proposed framework.\\
\end{frame}

\logo{}

\begin{frame}
	\frametitle{Major Reviewers' Comments}
	
	\large
	
	\textbf{Rev.}\\
	``For some experiments, comparisons are only made with various versions of the same algorithms.''
	
	\vspace{0.3cm}
	
	\textbf{Rev.}\\
	``The benefits of the method are shown for activity forecasting applications, intention prediction,
	and for constructing interactive costmaps to guide robot navigation. The latter applications
	represent significant contributions in robotics. Some additional discussion of the assumptions being  
	employed would be useful. Specifically, the joint optimization seems to assume more coordination
	than, e.g., humans have when they navigate (often sub-­optimally).'' \\
\end{frame}

\logo
{
	\begin{tikzpicture}
		\hspace{0.09cm}
		\node at (-1,0) [draw=white,ultra thick,inner sep=0pt]
		{
			\includegraphics[height=1cm]{ThemeFigs/Sapienza}
		};
		\node at (0,0) [draw=white,ultra thick,inner sep=0pt]
		{
			\includegraphics[height=1cm]{ThemeFigs/Edinburgh}
		};
	\end{tikzpicture}
	\vspace{199.1pt}
}

\begin{frame}
	\frametitle{Full List of Publications}
	
	\vspace{-0.2cm}
	
	\begin{columns}[t]
		\column{0.48\textwidth}
		
		\tiny
		
		\textcolor{red}{\textbf{\underline{Journal}}}
		
		\vspace{0.1cm}
		
		\textbf{1. Multi-Robot Surveillance through a Distributed Sensor Network}, A. Pennisi, F.
		Previtali, C. Gennari, D. D. Bloisi, L. Iocchi, F. Ficarola, A. Vitaletti, D. Nardi \\
		\emph{Journal of Studies in Computational Intelligence, 2015}
		
		\vspace{0.15cm}
		
		\textbf{2. Distributed Sensor Network for Multi-Robot Surveillance}, A. Pennisi, F. Previtali,
		F. Ficarola, D. D. Bloisi, L. Iocchi, A. Vitaletti \\
		\emph{Procedia Computer Science, 2014}
		
		\vspace{0.2cm}
		
		\textcolor{red}{\textbf{\underline{Conference}}}
		
		\vspace{0.1cm}
		
		\textbf{3. PTracking: Distributed Multi-Agent Multi-Object Tracking through Multi-Clustered
		Particle Filtering}, F. Previtali, L. Iocchi \\
		\emph{IEEE Conference on Multisensor Fusion and Integration, 2015}
		
		\vspace{0.15cm}
		
		\textbf{4. Disambiguating Localization Symmetry through a Multi-Clustered Particle Filtering},
		F. Previtali, G. Gemignani, L. Iocchi, D. Nardi \\
		\emph{IEEE Conference on Multisensor Fusion and Integration, 2015}
		
		\vspace{0.15cm}
		
		\textbf{5. Counterfactual Reasoning about Intent for Interactive Navigation in Dynamic
		Environments}, A. Bordallo, F. Previtali, N. Nardelli, S. Ramamoorthy \\
		\emph{IEEE Conference on Intelligent Robots and Systems, 2015}
		
		\vspace{0.15cm}
		
		\textbf{6. IRL-based Prediction of Goals for Dynamic Environments}, F. Previtali, A. Bordallo,
		S. Ramamoorthy \\
		\emph{Machine Learning for Social Robotics at ICRA, 2015}
		
		\column{0.53\textwidth}
		
		\vspace{0.34cm}
		
		\tiny
		
		\textbf{7. Real-Time Adaptive Background Modeling in Fast Changing Conditions}, A. Pennisi, F.
		Previtali, D. D. Bloisi, L. Iocchi \\
		\emph{Conference on Advanced Video and Signal based Surveillance, 2015}
		
		\vspace{0.2cm}
		
		\textcolor{red}{\textbf{\underline{Submitted}}}
		
		\vspace{0.1cm}
		
		\textbf{8. A Distributed Approach for Real-Time Multi-Camera Multi-Object Tracking}, F.
		Previtali, D. D. Bloisi, L. Iocchi \\
		\emph{Journal on Machine Vision and Applications}
		
		\vspace{0.15cm}
		
		\textbf{9. Enhancing Automatic Maritime Surveillance Systems with Visual Information}, D. D.
		Bloisi, F. Previtali, A. Pennisi, D. Nardi, M. Fiorini \\
		\emph{Transaction on Intelligent Transportation Systems}
		
		\vspace{0.15cm}
		
		\textbf{10. Predicting Future Agent Motions for Dynamic Environments}, F. Previtali, A.
		Bordallo, L. Iocchi, S. Ramamoorthy \\
		\emph{IEEE Conference on Intelligent Robots and Systems}
		
		\vspace{0.15cm}
		
		\textbf{11. Interactive Costmaps: Integrating Prediction and Planning with Counterfactual
		Reasoning}, A. Bordallo, F. Previtali, S. Ramamoorthy \\
		\emph{IEEE Conference on Intelligent Robots and Systems}
	\end{columns}
\end{frame}

\logo{}

\begin{frame}
	\frametitle{That's all!}
	
	\begin{figure}[!h]
		\centering
		\includemovie[inline=false,text=
		{
			\begin{tikzpicture}
				\node at (0,0) [draw=white,ultra thick,inner sep=0pt]
				{
					\includegraphics[width=0.8\linewidth]{Figures/Questions}
				};
			\end{tikzpicture}
		}]{}{}{../Videos/2-Nao/PTracking-RLSD.avi}
	\end{figure}
\end{frame}


\tiny
\bibliographystyle{plain}
\bibliography{Main}

\end{document}
