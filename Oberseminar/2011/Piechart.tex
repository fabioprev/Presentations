\usepackage{calc}
\usepackage{ifthen}

% Pie chart
% Author: Robert Vollmert
\newcounter{temp}


% calculate the anchor of the external label
\newcommand{\angledir}[1]{
  \setcounter{temp}{#1}

  \ifthenelse{\thetemp < 0}{\addtocounter{temp}{360}}{}
  \ifthenelse{\thetemp > 360}{\addtocounter{temp}{-360}}{}

  \ifthenelse{\thetemp < 11}{\def\piedir{right}}{
  \ifthenelse{\thetemp < 80}{\def\piedir{above right}}{
  \ifthenelse{\thetemp < 101}{\def\piedir{above}}{
  \ifthenelse{\thetemp < 170}{\def\piedir{above left}}{
  \ifthenelse{\thetemp < 191}{\def\piedir{left}}{
  \ifthenelse{\thetemp < 260}{\def\piedir{below left}}{
  \ifthenelse{\thetemp < 281}{\def\piedir{below}}{
  \ifthenelse{\thetemp < 350}{\def\piedir{below right}}{
    right}}}}}}}}%
}

% calculate the position of the internal label
\newcommand{\calcpiedist}[1]{%
  \ifthenelse{#1 > 120}{\def\piedist{0.50}}{
  \ifthenelse{#1 < 10}{\def\piedist{0.80}}{
    \setcounter{temp}{(80 * (120-#1) + 50 * (#1-10))/110}
    \def\piedist{0.\thetemp}}}
}

% draw a slice of a pie chart
% The diffI and diffII counters are just a quick hack for making the
% code compatible with PGF 1.18
% The code should be rewritten to utilize the new math engine.
\newcounter{diffI}
\newcounter{diffII}
\newcommand{\slice}[5]{%
  \setcounter{temp}{(#1+#2)/2}
  \setcounter{diffI}{#1-\thetemp}
  \setcounter{diffII}{#2-\thetemp}
  \begin{scope}[rotate=\thetemp]
  %\pgfmathparse{#1-\thetemp}
    \draw[fill=#5,join=round]
                                (0,0)
                             -- (\thediffI:1)
                             arc (\thediffI:\thediffII:1)
                             -- cycle;
    \angledir{\thetemp}
    \node [\piedir] at (1,0) {#4};
    \setcounter{temp}{#2-#1}
    \calcpiedist{\thetemp}
    \node at (\piedist,0) {#3};

  \end{scope}
}

